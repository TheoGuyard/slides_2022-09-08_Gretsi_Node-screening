\documentclass[10pt]{beamer}

\usetheme[
  numbering=fraction,
  block=fill,
  % background=dark
  ]{metropolis}

\usepackage{appendixnumberbeamer}
\usepackage{booktabs}
\usepackage[scale=2]{ccicons}
%\usepackage{pgfplots}
%\usepgfplotslibrary{dateplot}

\usepackage[absolute,overlay]{textpos}
\setlength{\TPHorizModule}{\paperwidth}
\setlength{\TPVertModule}{\paperheight}

% ------------------------ %
% 	  Packages		%
% ------------------------ %

\usepackage[utf8]{inputenc}
\usepackage[T1]{fontenc}
\usepackage{xcolor}
\usepackage[setbb,setcolon]{kmath}
\usepackage{xparse,pgffor,ifthen}
\usepackage{amsmath,amssymb,amsfonts,amsthm}
\usepackage{mathrsfs,mathtools,bbm}
\usepackage{booktabs,dcolumn,multirow,stmaryrd,array}
\newcolumntype{d}{D{.}{.}{3.2}}
\newcolumntype{e}{D{.}{.}{1.2}}
\usepackage{pgfplots,forest}
\pgfplotsset{compat=newest}
\usepgfplotslibrary{groupplots}
\usepackage{pgfplotstable}
\usepackage[ruled,linesnumbered]{algorithm2e}
\usepackage{cleveref,csquotes,comment,lipsum,todonotes,soul,extarrows,graphicx}
\usepackage{glossaries}
\usepackage[numbers,sort&compress]{natbib}
\usepackage{glossaries,url,subcaption}
\usepackage[texcoord]{eso-pic}
\include{headings/shortcuts}
% ------------------------ %
% 			Macros		   %
% ------------------------ %

% Fonts
\newcommand{\fontset}[1]{\mathcal{#1}}
\newcommand{\fontnode}[1]{\text{\scshape#1}}
\newcommand{\fontoptpb}[1]{#1}

% Data
\newcommand{\datafunc}{f}
\newcommand{\pdim}{n}
\newcommand{\ddim}{m}
\newcommand{\obs}{\y}
\newcommand{\dic}{\A}
\newcommand{\atom}{\a}
\newcommand{\reg}{\lambda}
\newcommand{\bigM}{M}
\newcommand{\pivot}[2]{\boldsymbol{\gamma}_{#1}^{#2}}

% Screening sets
% \newcommand{\nodeSymb}{\fontnode{n}}
\newcommand{\nodeSymb}{\nu}
\newcommand{\nodeSymbIter}[1]{\nodeSymb^{(#1)}}
\newcommand{\setnodesymb}{\fontset{S}}
\newcommand{\setzero}{\setnodesymb_0}
\newcommand{\setone}{\setnodesymb_1}
\newcommand{\setnone}{\bar{\setnodesymb}}

\newcommand{\nodePlusZero}[2]{#1\cap\{\pve_{#2}=0\}}
\newcommand{\nodePlusOne}[2]{#1\cap\{\pve_{#2}\neq0\}}

% Superscripts and subscripts
\newcommand{\subzero}[1]{#1_{\setzero}}
\newcommand{\subone}[1]{#1_{\setone}}
\newcommand{\subnone}[1]{#1_{\setnone}}
\newcommand{\node}[1]{#1^{\nodeSymb}}

% Upper/Lower bounds on optimal values
\newcommand{\UB}[1]{\bar{#1}}
\newcommand{\LB}[1]{\tilde{#1}}

% Problems
\newcommand{\primalletter}{\fontoptpb{P}}
\newcommand{\dualletter}{\fontoptpb{D}}
\newcommand{\mpb}{\primalletter}
\newcommand{\spb}{\primalletter}
\newcommand{\hpb}{\UB{\primalletter}}
\newcommand{\rpb}{\LB{\primalletter}}
\newcommand{\dpb}{\dualletter}

% Objective values
\newcommand{\pobj}{p}
\newcommand{\optobj}{\opt{\pobj}}
\newcommand{\sobj}{\pobj}
\newcommand{\hobj}{\UB{\sobj}}
\newcommand{\robj}{\LB{\sobj}}
\newcommand{\dobj}{d}

% Objective functions
\newcommand{\pfunc}{\Prm}
\newcommand{\hfunc}{\UB{\pfunc}}
\newcommand{\rfunc}{\LB{\pfunc}}
\newcommand{\dfunc}{\Drm}
		
% Variables
\newcommand{\pve}{x}
\newcommand{\bve}{z}
\newcommand{\dve}{u}
\newcommand{\pv}{\mathbf{\pve}}
\newcommand{\pvrelax}{\mathbf{\pve}_l}
\newcommand{\bv}{\mathbf{\bve}}
%\newcommand{\dv}{\mathbf{\dve}_l}
\newcommand{\dv}{\mathbf{\dve}}
\newcommand{\dvopt}{\dv^\star_l}

\newcommand{\idxentry}{i}
\newcommand{\idxentrynode}{i}
\newcommand{\idxscreen}{\ell}

% Operators
\newcommand{\argmin}{\mathrm{argmin}}
\newcommand{\argmax}{\mathrm{argmax}}
\newcommand{\conj}[1]{#1^*}
\newcommand{\card}[1]{|#1|}
\newcommand{\grad}{\nabla}
\newcommand{\Icvx}{\eta}
\newcommand{\intervint}[2]{\llbracket#1,#2\rrbracket}
\newcommand{\norm}[2]{\|#1\|_#2}
\newcommand{\opt}[1]{#1^{\star}}
\newcommand{\scalprod}[2]{\langle{#1,#2}\rangle}
\newcommand{\sign}[1]{\mathrm{sign}({#1})}
\newcommand{\relu}[1]{[#1]_+}

\DeclarePairedDelimiter\ceil{\lceil}{\rceil}
\DeclarePairedDelimiter\floor{\lfloor}{\rfloor}

\include{headings/glossaries}

\title{Node-screening pour le problème des moindres carrés avec pénalité $\ell_0$}
\date{\textbf{GRETSI -- 8 Septembre 2022}}
\author{Théo Guyard${}^{1,2}$, Ayse-Nur Arslan${}^1$, Cédric Herzet${}^{2}$, Clément Elvira${}^{3}$ \\ \scriptsize{${}^{1}$ INSA Rennes} \\ \scriptsize{${}^{2}$ INRIA Rennes Bretagne Atlantique} \\ \scriptsize{${}^{3}$ IETR CentraleSupélec}}

\begin{document}

\begin{frame}
  \maketitle
\end{frame}

\section{$\ell_0$-penalized problems}

\begin{frame}{Sparse problem}
  \textbf{Ingredients of the problem}
  \begin{itemize}
    \item A \emphone{target} $\obs$
    \pause
    \item A \emphone{dictionary} $\dic=\{\atom_{\idxentry}\}_{\idxentry\in\mathcal{I}}$ made of \emphone{atoms} 
  \end{itemize}
  \pause
  \textbf{Objective}
  \begin{itemize}
    \item Find a \emphone{sparse} linear combination of atoms that \emphone{well approximates} the target through a given model
  \end{itemize}
  \pause
  \textbf{Rough formulation}
  \begin{center}
    \begin{minipage}{0.6\linewidth}
      \begin{block}{Problem}
        \centering
        Find $\pv$ \emphone{sparse} such that \emphone{$\obs \simeq$ Model$(\dic \pv)$}
      \end{block}
    \end{minipage}
  \end{center}
  \pause
  Remark : Entries of $\pv$ weight each atom in the linear combination.
\end{frame}


\begin{frame}{$\ell_0$-penalized problem}
  \begin{block}{$\ell_0$-penalized problem}
    \begin{equation}
      \label{prob:prob} \tag{\(\mpb\)}
      \optobj \ =
      \left\{
      \begin{array}{rl}
        \min & \datafunc(\dic\pv) + \reg \norm{\pv}{0}\\
        \text{s.t.} & \kvvbar{\pv}_{\infty} \leq \bigM
      \end{array}
      \right.
    \end{equation}
    where $\reg > 0$ is a tuning parameter and $\bigM$ is a big-enough constant.
  \end{block}

  \pause

  \begin{center}
    Problem \eqref{prob:prob} $\quad \xlongrightarrow{\text{reformulation}} \quad$\glsdesc{mip}
  \end{center}

  \pause

  \textbf{Properties :}
  \begin{itemize}
    \item Continuous and integer variables
    \item Combinatorial problem
    \item Can be addressed with \emphone{\glsdesc{bnb} (\gls{bnb})} algorithms
  \end{itemize}
\end{frame}

\section{Branch-and-bound algorithms}

\begin{frame}{Branch-and-bound principle}
  \textbf{Idea :}
  \begin{itemize}
    \item Enumerate all feasible solutions
    \pause
    \item Use tests to discard irrelevant candidates
    \pause
    \item[$\rightarrow$] \emphone{Explore a decision tree and prune uninteresting nodes} 
  \end{itemize}

  \pause

  \textbf{Node} $\nodeSymb = (\setzero,\setone,\setnone)$ \textbf{where :}
  \begin{itemize}
    \item $\setzero$ : indices of $\pv$ fixed to \emphone{zero}
    \item $\setone$ : indices of $\pv$ fixed to \emphone{non-zero}
    \item $\setnone \ $ : indices not fixed yet
  \end{itemize}

  \pause

  \begin{figure}
    \centering
    \scalebox{0.7}{\forestset{
    node/.style = {
        draw, 
        circle,
        thick,
        align = center,
        font = \small,
        top color = white,
        bottom color = blue!25,
    },
    branch label/.style={
        edge label = {
            node[midway,fill=white,font=\small,draw=black,text height=0.5em,scale=0.9]{#1}
        }
    },
    bnb/.style={
        branch label,
        for tree = {
            node,
            s sep'+=14mm,
            l sep'+=2mm,
            edge={->},
            edge+={thick},
        },
        before typesetting nodes={
            for tree={
                split option={content}{:}{content,branch label},
            },
        },
        where n children=0{
            tikz+={
                \draw [thick,dashed,->]  ([yshift=0pt, xshift=0pt].south east) -- ([yshift=-5pt, xshift=5pt].south east);
                \draw [thick,dashed,->]  ([yshift=0pt, xshift=0pt].south west) -- ([yshift=-5pt, xshift=-5pt].south west);
            }
        }{},
    },
}
\begin{forest}
    bnb,
    [
        \(\nodeSymbIter{0}\)
            [\(\nodeSymbIter{1}\):\({\pve_{\idxentrynode_1}=0}\)
            [\(\nodeSymbIter{3}\):\({\pve_{\idxentrynode_2}=0}\)]
            [\(\nodeSymbIter{4}\):\({\pve_{\idxentrynode_2}\neq0}\)]
    ][
        \(\nodeSymbIter{2}\):\({\pve_{\idxentrynode_1}\neq0}\),
            [\(\nodeSymbIter{5}\):\({\pve_{\idxentrynode_2}=0}\)]
            [\(\nodeSymbIter{6}\):\({\pve_{\idxentrynode_2}\neq0}\)]
    ]
    ]
\end{forest}}
  \end{figure}
\end{frame}

\begin{frame}{Processing node $\nodeSymb = (\setzero,\setone,\setnone)$}

  \textbf{Question :} Does any solution of \eqref{prob:prob} match the current constraints ? 

  \pause
  \vspace{0.2cm}

  \begin{block}{\emphone{Sub}-problem at node $\nodeSymb$}
    \begin{equation}
      \label{prob:prob-node} \tag{\(\node{\mpb}\)}
      \node{\pobj} \ =
      \left\{
      \begin{array}{rl}
        \min & \datafunc(\dic\pv) + \reg \norm{\pv}{0}\\
        \text{s.t.} & \kvvbar{\pv}_{\infty} \leq \bigM
      \end{array}
      \right\}
      \bigcap
      \emphone{
      \left\{
      \begin{array}{rl}
        \subzero{\pv} &= \0 \\
        \subone{\pv} &\neq \0
      \end{array}
      \right\}
      }
    \end{equation}
    If \emphone{$\opt{\pobj} < \node{\pobj}$}, then node $\nodeSymb$ can be \emphone{pruned} from the \gls{bnb} tree.
  \end{block}

  \pause
  Neither $\opt{\pobj}$ nor $\node{\pobj}$ are accessible in practice :
  \pause
  \begin{itemize}
    \item \emphone{Restriction} : Upper bound $\UB{\pobj}$ on $\opt{\pobj}$ \\
    \pause
    \item \emphone{Relaxation} : Lower bound $\node{\robj}$ on $\node{\pobj}$ \\
    \pause
    \item If \emphone{$\UB{\pobj} < \node{\robj}$}, then node $\nodeSymb$ can be \emphone{pruned}
    \pause 
    \item Both $\UB{\pobj}$ and $\node{\robj}$ are computed by solving convex problems
  \end{itemize}

  \pause
  \begin{center}
    \input{img/relaxation}
  \end{center}

\end{frame}

\begin{frame}{Exploration and pruning process}
  \begin{figure}
    \centering
    \scalebox{1}{\forestset{
    node/.style = {
        draw, 
        circle,
        thick,
        align = center,
        font = \small,
        top color = white,
        bottom color = blue!25,
    },
    branch label/.style={
        edge label = {
            node[midway,fill=white,font=\small,draw=black,text height=0.5em,scale=0.9]{#1}
        }
    },
    bnb/.style={
        branch label,
        for tree = {
            node,
            s sep'+=14mm,
            l sep'+=2mm,
            edge={->},
            edge+={thick},
        },
        before typesetting nodes={
            for tree={
                split option={content}{:}{content,branch label},
            },
        },
        where n children=0{
            tikz+={
                \draw [thick,dashed,->]  ([yshift=0pt, xshift=0pt].south east) -- ([yshift=-5pt, xshift=5pt].south east);
                \draw [thick,dashed,->]  ([yshift=0pt, xshift=0pt].south west) -- ([yshift=-5pt, xshift=-5pt].south west);
            }
        }{},
    },
}
\begin{forest}
    bnb,
    [
        \(\nodeSymbIter{0}\)
            [\(\nodeSymbIter{1}\):\({\pve_{\idxentrynode_1}=0}\)
            [\(\nodeSymbIter{3}\):\({\pve_{\idxentrynode_2}=0}\)]
            [\(\nodeSymbIter{4}\):\({\pve_{\idxentrynode_2}\neq0}\)]
    ][
        \(\nodeSymbIter{2}\):\({\pve_{\idxentrynode_1}\neq0}\),
            [\(\nodeSymbIter{5}\):\({\pve_{\idxentrynode_2}=0}\)]
            [\(\nodeSymbIter{6}\):\({\pve_{\idxentrynode_2}\neq0}\)]
    ]
    ]
\end{forest}}
  \end{figure}
\end{frame}

\begin{frame}{Exploration and pruning process}
  \begin{figure}
    \centering
    \scalebox{1}{\input{img/bnb-2}}
  \end{figure}
\end{frame}

\begin{frame}{\gls{bnb} efficiency}
  The efficiency of the \gls{bnb} algorithm depends on :
  \begin{itemize}
    \item The number of nodes processed
    \item The ability to process nodes quickly
  \end{itemize}
  Node-screening improves both of these things !
\end{frame}

\section{Node-screening}

\begin{frame}{Main idea}

  \begin{center}
    Testing if a node can be pruned $\quad\equiv\quad$ Solving convex problems
  \end{center}
  
  \begin{textblock*}{100pt}(230pt,110pt)
    \begin{center}
      \scriptsize{\textcolor{red}{
        $\uparrow$ \\
        This is costly      
      }}
    \end{center}
  \end{textblock*}

  \begin{textblock*}{100pt}(45pt,110pt)
    \begin{center}
      \scriptsize{\textcolor{red}{
        $\uparrow$ \\
        Sometimes it is obvious !
      }}
    \end{center}
  \end{textblock*}

  \vspace*{1cm}

  \pause
  \textbf{Question :} How to detect prunable nodes in a more \emphone{economic} way ?
\end{frame}

\begin{frame}{Dual problem}

  \begin{center}
    \forestset{
    node/.style = {
        draw, 
        circle,
        thick,
        align = center,
        font = \small,
        top color = white,
        bottom color = blue!25,
    },
    branch label/.style={
        edge label = {
            node[midway,fill=white,font=\small,draw=black,text height=0.5em,scale=0.9]{#1}
        }
    },
    bnb/.style={
        branch label,
        for tree = {
            node,
            s sep'+=14mm,
            l sep'+=2mm,
            edge={->},
            edge+={dashed,thick},
        },
        before typesetting nodes={
            for tree={
                split option={content}{:}{content,branch label},
            },
        },
    },
}
\begin{forest}
    bnb,
    [
        \(\nodeSymbIter{0}\)
        [
            \(\nodeSymbIter{k}\)
        ]
    ]
\end{forest}
  \end{center}
  \begin{textblock*}{100pt}(205pt,85pt)
    \scriptsize{
      Node $\nodeSymbIter{k} = (\setzero,\setone,\setnone)$ \\
      $\bullet$ Sub-problem \\
      $\bullet$ Relaxed problem
    }
  \end{textblock*}

  \pause
  
  \begin{block}{\emphone{Dual} problem at node $\nodeSymb$}
    \begin{equation} 
      \label{prob:dual-node}
      \tag{$\node{\dpb}$} 
      \max_{\dv \in \kR^{\ddim}} \ 
      \Big\{ 
          \node{\dfunc}(\dv) \triangleq -\conj{\datafunc}(-\dv) - 
          \sum_{\emphone{\idxentry \in \setnone}} [\pivot{}{}(\ktranspose{\atom}_{\idxentry}\dv)]_+ - 
          \sum_{\emphone{\idxentry \in \setone}} \pivot{}{}(\ktranspose{\atom}_{\idxentry}\dv)
      \Big\}
    \end{equation}
  \end{block} 

  \pause

  \begin{itemize}
    \item One common term to all nodes
    \item Terms depending on the current node constraints
    \pause
    \item The \emphone{pivot} function is defined as $\pivot{}{}(t) = \bigM |t| - \reg$
  \end{itemize}
\end{frame}

\begin{frame}{Dual objective link}
  \textbf{Direct consequence :} The dual objective at two consecutive nodes differs from only one term.

  \pause
  \vspace{0.2cm}
  \begin{block}{Dual objective link}
    At node $\nodeSymb=(\setzero,\setone,\setnone)$, let $\idxentry \in \setnone$. Then $\forall \dv$, 
    \begin{subequations}
        \begin{align*}
            \dfunc^{\emphone{\nodePlusZero{\nodeSymb}{\idxentry}}}(\dv) &= \dfunc^{\emphone{\nodeSymb}}(\dv) + [\pivot{}{}(\ktranspose{\atom}_{\idxentry}\dv)]_+
            \\
            \dfunc^{\emphone{\nodePlusOne{\nodeSymb}{\idxentry}}}(\dv) &= \dfunc^{\emphone{\nodeSymb}}(\dv) - [\pivot{}{}(\ktranspose{\atom}_{\idxentry}\dv)]_-
        \end{align*}
    \end{subequations}
  \end{block}

  \pause
  \begin{center}
    \begin{tikzpicture}
    \draw[thick,->] (0,0) -- (6,0);
    \node at (6.5,0.05) (axis_text) {value};

    \node at (2,0) (pv1) {};
    \draw[thick,teal] (pv1.north) -- (pv1.south);
    \node[teal] at (2,-0.4) (pv1_text) {$\node{\robj}$};

    \pause
    \node at (1,0) (dv1) {};
    \draw[thick,teal] (dv1.north) -- (dv1.south);
    \node[teal] at (1,-0.4) (dv1_text) {$\node{\dfunc}(\dv)$};
    \draw[thick,teal,->] (pv1) .. controls (1.5,0.5) and (1.5,0.5) .. (dv1);
    \node[teal] at (1.5,0.6) (dv1_arrow_text) {\scriptsize dualize};

    \pause
    \node at (4,0) (dv2) {};
    \draw[thick,orange] (dv2.north) -- (dv2.south);
    \node[orange] at (4,-0.4) (dv2_text) {$\dfunc^{\nodePlusZero{\nodeSymb}{\idxentry}}(\dv)$};
    \draw[thick,orange,->] (dv1_text.south) .. controls (2.5,-1.5) and (2.5,-1.5) .. (dv2_text.south);
    \node[orange] at (2.5,-1.6) (dv1_arrow_text) {\scriptsize $+[\pivot{}{}(\ktranspose{\atom}_{\idxentry}\dv)]_+$};

    \pause
    \node at (2.5,0) (ub) {};
    \draw[thick,red] (ub.north) -- (ub.south);
    \node[red] at (2.5,-0.4) (ub_text) {$\UB{\pobj}$};

\end{tikzpicture}
  \end{center}
\end{frame}

\begin{frame}{Node-screening test}

  \begin{block}{Node-screening test}
    Given some point $\dv$, 
    \begin{alignat*}{4}
			\node{\dfunc}(\dv) + [\pivot{}{}(\ktranspose{\atom}_{\idxentry}\dv)]_+ & > \UB{\pobj} & \quad\implies\quad & \emphone{\text{Fix} \ \pve_{\idxentry} \neq 0 \ \text{at node} \ \nodeSymb} \\
			\node{\dfunc}(\dv) - [\pivot{}{}(\ktranspose{\atom}_{\idxentry}\dv)]_- & > \UB{\pobj} & \quad\implies\quad & \emphone{\text{Fix} \ \pve_{\idxentry} = 0 \ \text{at node} \ \nodeSymb}
		\end{alignat*}
  \end{block}

  \pause

  \textbf{Nesting property :} If \emphone{multiple} node-screening tests are passed, the corresponding variables can be fixed \emphone{simultaneously}.
\end{frame}

\begin{frame}{Consequence of passing a node-screening test}
  \begin{figure}
    \centering
    \scalebox{1}{\forestset{
    node/.style = {
        draw, 
        circle,
        thick,
        align = center,
        font = \small,
        top color = white,
        bottom color = blue!25,
    },
    branch label/.style={
        edge label = {
            node[midway,fill=white,font=\small,draw=black,text height=0.5em,scale=0.9]{#1}
        }
    },
    bnb/.style={
        branch label,
        for tree = {
            node,
            s sep'+=14mm,
            l sep'+=2mm,
            edge={->, thick, draw opacity=0.2, text opacity = 0.3},
            opacity = 0.25,
            text opacity = 0.25,
        },
        before typesetting nodes={
            for tree={
                split option={content}{:}{content,branch label},
            },
        },
        where n children=0{
            tikz+={
                \draw [thick,dashed,->,opacity=0.25] ([yshift=0pt, xshift=0pt].south east) -- ([yshift=-5pt, xshift=5pt].south east);
                \draw [thick,dashed,->,opacity=0.25]  ([yshift=0pt, xshift=0pt].south west) -- ([yshift=-5pt, xshift=-5pt].south west);
            }
        }{},
    },
}
\begin{forest}
    bnb,
    [\(\nodeSymbIter{0}\),name=root,opacity=1,text opacity=1
        [\(\nodeSymbIter{1}\):\({\pve{\idxentrynode_1}=0}\)
            [\(\nodeSymbIter{3}\):\({\pve{\idxentrynode_2}=0}\),name=S3,opacity=1,text opacity=1] {
                \draw [thick,dashed,->] ([yshift=0pt, xshift=0pt].south east) -- ([yshift=-5pt, xshift=5pt].south east);
                \draw [thick,dashed,->]  ([yshift=0pt, xshift=0pt].south west) -- ([yshift=-5pt, xshift=-5pt].south west);
                \draw[thick,->] (root.west) .. controls (-3.8,0) .. (S3.north);
                \node[draw,thick,top color = white,
                bottom color = orange!25,,font=\scriptsize,align=center] at (-3.3,-0.4) (scr) {\bf Node-screening \\ Fix \({\pve_{\idxentrynode_1}=0}\) \\ Fix \({\pve_{\idxentrynode_2}=0}\)};
            }
            [\(\nodeSymbIter{4}\):\({\pve{\idxentrynode_2}\neq0}\)]
        ]
        [\(\nodeSymbIter{2}\):\({\pve{\idxentrynode_1}\neq0}\),
            [\(\nodeSymbIter{5}\):\({\pve{\idxentrynode_2}=0}\)]
            [\(\nodeSymbIter{6}\):\({\pve{\idxentrynode_2}\neq0}\)]
        ]
    ]
\end{forest}}
  \end{figure}
  \pause
  \textbf{Consequence :} Less nodes are explored by the \gls{bnb} algorithm.
\end{frame}

\section{Some numerical results}

\begin{frame}{Some numerical results}
  \newcommand{\CPLEX}{\texttt{CPLEX}}
  \newcommand{\BNB}{\texttt{BnB}}
  \newcommand{\BNBscr}{\texttt{BnB+scr}}
  \newcommand{\sparsitylevel}{k}

  \textbf{Synthetic setups :} 
  \begin{enumerate}
    \item Sample a synthetic $\dic \in \kR^{\ddim\times\pdim}$ with $(\ddim,\pdim)=(500,1000)$ 
    \pause
    \item Generate a $\sparsitylevel$-sparse vector $\opt{\pv}$ with $\sparsitylevel \in \{5,\dots,20\}$
    \pause
    \item Set $\obs = \dic \opt{\pv}+$ noise with $10$dB SNR
    \pause
    \item Tune $\reg$ and $\bigM$ to (hopefully) recover $\opt{\pv}$ by solving \eqref{prob:prob}
  \end{enumerate}

  \pause

  \vspace*{0.25cm}
  \begin{figure}
    \centering
    \begin{minipage}[t]{0.49\textwidth}
      \begin{tikzpicture}
    \begin{axis}[
        width   = \linewidth,
        height  = \linewidth,
        xlabel  = $\sparsitylevel$,
        ylabel  = Node count,
        legend style={
            at={(0.02,0.98)}, 
            anchor=north west
        },
    ]

        \legend{\texttt{BnB},\texttt{BnB+scr}}
        
        \pgfplotstableread[col sep=comma]{img/m=500_n=1000_rho=0.1_s=10dB.dat}{\data}
        
        \addplot[
            blue,
            line width=2pt
        ] table [x={k}, y={bnb-l1_node_count_mean}] {\data};

        \addplot[
            orange,
            line width=2pt
        ] table [x={k}, y={bnb-l0l1_node_count_mean}] {\data};

        \end{axis}
\end{tikzpicture}
    \end{minipage}
    \hfill
    \pause
    \begin{minipage}[t]{0.49\textwidth}
      \input{img/perf-time}
    \end{minipage}
  \end{figure}
\end{frame}

\end{document}
